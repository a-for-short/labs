\documentclass{article}
\usepackage{booktabs} % for professional looking tables
\usepackage{array}
\usepackage[utf8]{inputenc}
\usepackage[russian]{babel}
\usepackage{siunitx}
\usepackage{csquotes}
\usepackage{url,amsmath,amssymb,fancybox,listings,pdfpages,caption,multicol,datetime,rotating, booktabs}
\usepackage{geometry}
\usepackage{graphicx}
\author{Николаев Владислав\\
	\texttt{v.nikolaev2@g.nsu.ru} \and
	Матяш Алексей\\
	\texttt{a.matyash@g.nsu.ru}}
\title{Определение константы диссоциации бромтимолового синего}
\begin{document}
	\maketitle
	
	\begin{table}[htbp]
		\centering
		\caption{Данные для индикатора при $\lambda = 615$ нм}
		\resizebox{\textwidth}{!}{ % Adjust table size to fit within the page
			\begin{tabular}{>{\centering\arraybackslash}m{3cm} >{\centering\arraybackslash}m{3cm} >{\centering\arraybackslash}m{3cm} >{\centering\arraybackslash}m{3cm} >{\centering\arraybackslash}m{4cm}}
				\toprule
				\textbf{Форма индикатора} & \textbf{Номер раствора} & \textbf{C (моль/л)} & \textbf{$D_{615}$} & \textbf{Коэффициент экстинкции} \\
				\midrule
				\textbf{HA$^-$} & 1 & $5.13 \times 10^{-5}$ & $1.91 \times 10^{-3}$ & 27 \\
				& 2 & $3.84 \times 10^{-5}$ & $1.43 \times 10^{-3}$ & \\
				& 3 & $2.56 \times 10^{-5}$ & $2.13 \times 10^{-4}$ & \\
				& 4 & $1.28 \times 10^{-5}$ & $3.37 \times 10^{-4}$ & \\
				\textbf{A$^{2-}$} & 5 & $2.56 \times 10^{-5}$ & $9.92 \times 10^{-1}$ & 36 411 \\
				& 6 & $1.28 \times 10^{-5}$ & $4.76 \times 10^{-1}$ & \\
				& 7 & $6.41 \times 10^{-6}$ & $2.08 \times 10^{-1}$ & \\
				& 8 & $3.20 \times 10^{-6}$ & $1.20 \times 10^{-1}$ & \\
				\bottomrule
			\end{tabular}
		} % End of resizebox
	\end{table}
	
	\vspace{1cm}
	
	\begin{table}[htbp]
		\centering
		\caption{Данные для индикатора при $\lambda = 433$ нм}
		\resizebox{\textwidth}{!}{
			\begin{tabular}{>{\centering\arraybackslash}m{3cm} >{\centering\arraybackslash}m{3cm} >{\centering\arraybackslash}m{3cm} >{\centering\arraybackslash}m{3cm} >{\centering\arraybackslash}m{4cm}}
				\toprule
				\textbf{Форма индикатора} & \textbf{Номер раствора} & \textbf{C (моль/л)} & \textbf{$D_{433}$} & \textbf{Коэффициент экстинкции} \\
				\midrule
				\textbf{HA$^-$} & 1 & $5.13 \times 10^{-5}$ & $9.07 \times 10^{-1}$ & 17 136 \\
				& 2 & $3.84 \times 10^{-5}$ & $6.66 \times 10^{-1}$ & \\
				& 3 & $2.56 \times 10^{-5}$ & $4.34 \times 10^{-1}$ & \\
				& 4 & $1.28 \times 10^{-5}$ & $2.12 \times 10^{-1}$ & \\
				\textbf{A$^{2-}$} & 5 & $2.56 \times 10^{-5}$ & $8.17 \times 10^{-2}$ & 2 987 \\
				& 6 & $1.28 \times 10^{-5}$ & $3.77 \times 10^{-2}$ & \\
				& 7 & $6.41 \times 10^{-6}$ & $1.42 \times 10^{-2}$ & \\
				& 8 & $3.20 \times 10^{-6}$ & $1.15 \times 10^{-2}$ & \\
				\bottomrule
			\end{tabular}
		} % End of resizebox
	\end{table}
	
\end{document}
